%%% -*-LaTeX-*-
% \renewcommand{\thesection}{\arabic{section}}

\chapter{Introduction}
% \providecommand*{\toclevel@groupheader}{0}
\label{chap:intro}

The problem this dissertation addresses is the persistent tension between representativeness and interactivity in visualizing a large number of events. As event data grows in scale, visualizations that are representative of events become less interactive, while systems optimized for interactivity risk discarding or misrepresenting events. The central thesis of this dissertation is that useful visual solutions require connecting visual design considerations with data management strategies. As an introduction to this thesis, this chapter describes event sequences, the motivation and challenges, and then outlines the dissertation thesis and its contributions.

\section{Event Sequence}
\label{sec:intro-event}
Event sequence data captures the relationships between events as they occur over time. This type of data is commonly found in fields such as healthcare, manufacturing, and computing. Analyzing event progression and relationships helps in decision-making that is difficult to make from individual events. One notable example of event sequence data analysis arises in High Performance Computing (HPC), where understanding the behavior of HPC applications in execution is essential in identifying performance bottlenecks, exploring opportunities for optimization, and planning for efficient resource utilization~\cite{isaacs2014state, davidson2023qualitative}. In healthcare, the health progression (events) of different patients are recorded, and clinicians analyze the data to make decisions on patient treatments and drug prescriptions~\cite{antweiler2022uncovering, bernard2018using}. Another example of event sequence data analysis is found in manufacturing, where project managers monitor different stages (events) of product development progress in multiple facilities.~\cite{lee2022linear, undozerov2021spring}. These examples highlight the importance of large and complex event sequence analysis for decision-making in diverse domains.

\begin{figure}[t]
  \centering
  \includegraphics[width=0.9\textwidth]{figs/gantt_depp.png}
  \caption{Visualizing interdependent event sequences using a Gantt chart (left) and a node-link diagram (right). Events are visualized using rectangular boxes placed over time (on the x-axis) and grouped by their location (on the y-axis) in parallel timelines. Event \textit{A} initiates event \textit{B}, which subsequently initiates three other events \textit{C, D, E} that make these events interdependent.}
  \label{fig:gantt_basics}
\end{figure}

Interactive visualization of event sequences plays an essential role in the exploratory analysis process. Raw event logs are visualized in interactive interfaces that allow users to identify patterns, detect anomalies, and reason about underlying events that inform the decision-making process. For example, execution traces generated from HPC applications contain a sequence of interdependent events. Interactive visual analysis of program execution traces generated from HPC applications makes it easier to understand runtime behavior and explore opportunities for achieving better execution performance. 

A wide variety of visual representations is available for visualizing event sequence data~\cite{yeshchenko2024survey, guo2021survey}. Gantt charts are a widely used idiom for visualizing interdependent event sequences. Conventionally, data in Gantt charts is organized along parallel timelines where the horizontal axis represents a notion of order (typically time). The vertical axis illustrates grouped events along different tracks (\autoref{fig:gantt_desc}). A rectangular bar in the chart represents an individual event or a summary of a set of events. A summary of a set of events is visualized using a rectangular bar when the number of events exceeds the number of visible pixels in the visual interface. The dependency between two events is visualized by a line connecting the dependent events.

Rectangular bar-based representations make event order, duration, overlap, gaps, and concurrency direct and legible. Additionally, the bar integrates naturally with statistical summaries that analysts rely on. For example, bar lengths are linear measures that can be aggregated into distributions (e.g., duration or histograms), binned over time to estimate utilization and occupancy, and grouped by color or category to compute proportions and rates. Compared to alternative representations, rectangular bars support a broad spectrum of low-level visual tasks, such as comparing events, highlighting outliers, and filtering~\cite{sakin2024gantttaxonomy}, which facilitates exploratory analysis.

\section{Motivation \& Challenges}
\label{sec:challenges}

This dissertation investigates the problem of designing a representative and interactive visualization of event sequences within the context of domains where automated event logs are generated over time. For example, during software performance analysis, a timestamped log of event traces is used for exploratory visual analysis.

Visualizing software traces with a large number of events for exploratory analysis has several challenges. The rectangular bar representation, along with the dependency lines, leads to the over-cluttering of the visual interface for a large number of events contained in the execution trace data. Over-cluttering occurs when overlapping bars and dense dependency lines visually obscure one another, making it difficult for users to distinguish individual events, follow dependencies, or identify meaningful patterns. Visual overload hampers the effectiveness of exploratory analysis and reduces the usability of the visualization.

HPC applications running for a longer time on a higher number of computing resources generally create execution traces with large numbers of events, ranging from hundreds of millions to billions of events. Apart from contributing to the over-cluttering issue, managing and visualizing a large number of interdependent events also contributes to larger visual latencies in typical interactions, such as zooming, panning, and filtering. Here, visual latency refers to the delay between when a user initiates one or more interactions and when the visualization system completes drawing the resulting data in the visual interface. Higher latency in the visual interactions hampers regular exploratory analysis tasks and reduces the usability of the visualization system~\cite{liu2014latency, davidson2023qualitative, battle2019role}.

Additionally, for a large number of events, a visualization system needs to generate summaries due to the limitation of available visible pixels and the requirement of reducing visual clutter. The summarization process, either with data aggregation or sub-sampling, must ensure that the summary does not misrepresent the underlying data, which could lead to incorrect interpretations during exploratory analysis. Furthermore, the summary should preserve the accuracy of key patterns and structures in the data while still being computationally efficient. A poorly designed summarization approach might omit critical event attribute values or distort relationships between events, reducing the effectiveness of the visualization for identifying trends, anomalies, or dependencies. Balancing interactivity with accurate representation remains a significant challenge in visualizing large execution traces~\cite{davidson2023qualitative, isaacs2014state}.

To maintain interactivity and accuracy with minimal memory consumption for visualization, data needs to be managed properly throughout the process, from when a user starts the interactions to when the drawing finishes. A data management technique dictates how user interactions are handled, and the underlying data are translated into visual representations using a framework. This framework comprises several key procedures and describes the relationship between the procedures. These procedures include identifying data queries for user interactions, fetching relevant data from the storage for the query, passing the data between different components (disk, main memory, or cache), processing the data based on the interaction requirements, preparing the data for drawing, and ultimately delivering the data for drawing to the visual interface. 

A poor data management technique lacks proper coordination among the procedures. For example, fetching data irrelevant to the query leads to unnecessary data processing, transfer, and filtering between the subsequent procedures. Then, repeated data reading between these procedures is redundant, as the data can be read once and reused by the subsequent procedures. Additionally, any mismatch in data formats among these procedures results in unnecessary conversions between different data formats. The lack of coordination between the procedures results in incrementally diminishing performance with higher interactive latency when the amount of data increases. Additionally, latency in any of these procedures contributes greatly to the latency of the whole framework and undermines the usability of the visualization system. Furthermore, poor data management leads to an increased memory footprint for visualization by storing data that is not even required to be visualized.

\section{Dissertation Thesis}
\label{sec:intro-claims}
The central thesis of this dissertation is that useful event sequence visualization requires connecting visual design considerations with data management techniques. Rather than treating visual design and data management as separate concerns, this work demonstrates that their interdependence is key to achieving both interactivity and representativeness. Visual design guidelines determine the principles of representing event sequences that are useful for the analysts in their specific use-cases, while data management strategies control what data can be interactively maintained at different scales. 
% By addressing these together, it is possible to design systems that are more useful than those that prioritize only one dimension.

\section{Contributions}
\label{sec:contribution}
To substantiate the central thesis, this dissertation integrates insights from three complementary research efforts, each addressing a different aspect of the visual designs and data management choices:

\begin{itemize}
    \item \textbf{Understanding design considerations to visualize the vast scale differences in HPC traces that are useful:} \textit{Traveler}~\cite{sakin2022traveler} is a design study to enable navigation on task parallel traces for performance analysis. Traveler presents techniques for navigating large HPC execution traces. It demonstrates that visual interfaces must account for the vast scale differences inherent in trace data while preserving meaningful detail at different levels. This work highlights the need for visual navigation strategies that balance visual clarity with computational scalability. (\autoref{chap:traveler})
    
    \item \textbf{Identifying visual tasks for Gantt charts and their relationship to data queries:} This work~\cite{sakin2024gantttaxonomy} contributes a literature-based visualization task taxonomy for Gantt Charts, which are widely used to represent event sequences. The taxonomy identifies the most frequent visual tasks and relates those to the underlying data query requirements. This work supports visualization design when a large number of events distributed across a high number of tracks necessitate interactions or additional views beyond the core chart. (\autoref{chap:gantt-taxonomy})
    
    \item \textbf{Developing data management strategies tailored to event sequence visualization balancing interactivity and visual accuracy:} \textit{ESeMan}~\cite{sakin2025managing} is a library for event sequence management for interactive visualization. It introduces a system that applies data summarization, hierarchical indexing, and retrieval strategies to event sequences. By aligning data management with the needs of visualization tasks, ESeMan demonstrates how interaction latency can be reduced without sacrificing the visual representativeness. This work illustrates the feasibility of managing event data in ways that serve interactive visualization directly. (\autoref{chap:eseman})
\end{itemize}

These contributions advance the central thesis of this dissertation by demonstrating how coordinated visualization design and related data management choices facilitate the development of useful event sequence visualization systems. These three contributions address distinct aspects of visualization design considerations and data management, and connect these through task taxonomies and data queries. This resulted in an integrated system with a focus on task parallel execution traces and Gantt chart representations.

These contributions also establish that effective event sequence visualization requires coordinated attention to both human factors and computational constraints. The research demonstrates how task taxonomies can systematically inform data management decisions, while illustrating the practical value of design study methodologies for developing scalable visualization systems. These contributions have implications beyond HPC, offering a methodological structure applicable to manufacturing, healthcare, and other domains where complex events and relationships within events require interactive visual analysis.

\section{Organization}
\label{sec:org}
The remainder of this dissertation is presented as follows: \autoref{chap:background} presents the background and related work on event sequence visualization. \autoref{chap:traveler}, \autoref{chap:gantt-taxonomy}, and \autoref{chap:eseman} discuss the primary contributions of this dissertation. \autoref{chap:discussion} presents discussions synthesizing insights from the three contributions, articulating design guidelines for connecting visual design with data management. A conclusion summarizing the contributions of this dissertation and outlining future research opportunities is presented in the \autoref{chap:conclusion}.