\chapter{Discussion}
\label{chap:discussion}

Designing useful visualizations for event sequences is challenging because of the persistent tension between representativeness and interactivity. As highlighted in~\autoref{chap:intro}, visualizations that aim to represent all events sacrifice interactivity, whereas systems designed for interactivity risk discarding or misrepresenting events. To ease this tension, in this dissertation, we claim \textbf{``useful visual solutions require treating visual design considerations and data management strategies as interconnected components, rather than separate concerns.''} This chapter provide discussions on how an interconnected visual design and data management strategies make useful visualizations for event sequences by easing the tension between representativeness and interactivity.

\section{Connecting Visual Design Considerations with Data Management Strategies}
% Through the course of this dissertation, we learned that effective visualization systems emerge from a tight coupling between visual design choices and underlying data management strategies. 
In this section, we discuss considerations for connecting visual design with data management strategies that emerged from our experiences and learned lessons. We synthesize the lessons learned across our studies and implementations, emphasizing how data abstraction, query optimization, and visualization strategies collectively influence interactivity and representativeness.

\subsection{Relating Visual Data Abstraction with Data Formats}
First, there needs to be a connection between data abstraction for the visualization design and a data format for managing the data. A data abstraction is the mapping of data to an abstract data type~\cite{munzner2014vad}. A data format is how data is stored and transmitted between different components, such as the visual interface, data query manager, and databases. Generally, visualization tools have a separate data abstraction that relates to the visual marks and rely on databases for storing the data. A mismatch between the data abstraction for visual design and the data format for management leads to the misrepresentation of data. Such a mismatch also increases the amount of data transfer and conversion needed between different visualization components. Relating data abstraction with a specific data format helps to reduce data misrepresentation, transfer, and conversion.

For example, during the design study of Traveler, we utilized a linear data format for visualization where each element of the linear structure maps into a visual mark: one to a darker box, zero to no box, and any value in between to a gray box. While storing the data, we utilized an interval tree. Therefore, prior to rendering the data, it was required to convert the interval tree into a linear structure to maintain our existing design choices. Later, we adopted a summed area table that gives a significant boost to the interactive performance, but that sacrifices accuracy. Our contribution with ESeMan adopted a hierarchical structure to control the accuracy, which also helps to tune the interactive performance by balancing the visual accuracy. The primary reduction in data query fetch time comes from fetching data directly, following the visual abstraction requirements, which reduces data conversion between different components while maintaining specific event attributes, such as start time or end time.

We followed a top-down approach, starting with choosing the visual data abstraction first and then adapting our data format to that abstraction. We modified the data management strategies that serve the pre-selected visual abstraction for the visual design. This ensured that design decisions about what to show and how to show it guided the organization and storage of the underlying data. While this top-down strategy proved effective in our research, other approaches remain open for investigation. A bottom-up approach, where existing data formats or structures dictate a feasible abstraction, may be necessary in contexts where legacy systems or performance-critical pipelines constrain the design space~\cite{kaldor2017canopy}. Hybrid strategies~\cite{ahrens2005paraview} that iteratively adjust between abstraction and format based on human feedback could balance flexibility with efficiency. Exploring when and how these alternative approaches lead to different trade-offs between representativeness, interactivity, and scalability is an important avenue for future research in event sequence visualization.

\subsection{Designing and Optimizing Queries Based on Visual Tasks}
\label{subsec:query_tasks}

A critical step in building event sequence visualization systems is to first identify the visual tasks that users need to perform. Establishing the task space provides a foundation for determining which queries are most important and where optimization will have the greatest impact. Moreover, recognizing that a single query can support multiple visual tasks creates opportunities to optimize both the data management and the visual interface. Optimizing such queries provides benefits in both improving interactivity and maintaining consistency across different visualization interfaces.

For example, in Traveler, both the Utilization view and the Gantt view are supported by the same query endpoint. This design allows us to compute and fetch data while serving multiple visual interfaces. As a result, interactivity is improved because the same optimized query serves multiple needs, and consistency is maintained because all views are based on a shared data representation. More generally, designing queries around identified visual tasks enables visualization systems to avoid redundancy in data retrieval, reduce discrepancies between coordinated views, and focus optimization efforts on the queries that matter most to visualization users.

\subsection{Visible Pixels Aware Data Abstraction}
\label{subsec:pixel_abstraction}

Another important design consideration is to anchor the visual data abstraction in the visible pixel space. A system can avoid unnecessary computation on data that cannot be distinguished at the current resolution. We adopt a pixel window approach, where the visible space is divided into pixel windows. Data management strategies are then adopted to fetch only the data necessary for these pixel windows.

This strategy provides two key benefits. First, it bounds the amount of data that must be processed to the display resolution, ensuring that interaction cost depends on the number of pixels rather than the total number of events. Second, it allows multiple summarization techniques, such as aggregation or sampling, to be naturally framed in terms of pixel windows, making results directly interpretable within the visual interface. In practice, this approach guided the design of both our query layer and supporting data structures, ensuring that interactivity is maintained even at large scales.

The principle of pixel-aware abstraction has been highlighted in prior visualization systems, such as imMens~\cite{liu2013immens}, Falcon~\cite{moritz2019falcon}, and Nanocubes~\cite{lins2013nanocubes}, all of which demonstrate that performance can be decoupled from raw data size by constraining queries to the visible resolution. Our work extends this principle to event sequence visualization, where queries aligned with pixel windows allow interactive navigation of timelines while preserving representativeness at the visual level. This pixel-based framing exemplifies the central thesis of the dissertation: useful visual solutions require connecting visual designs with data management strategies, here by binding queries and data structures directly to the constraints of the visual interface.


\subsection{Balancing Representativeness and Interactivity}
\label{subsec:balance}

Achieving both representativeness and interactivity simultaneously remains a difficult challenge for event sequence visualization. While our systems demonstrate that progress is possible, in practice, designers often need to balance between these two goals depending on user needs. One promising strategy is to provide options that allow users to control the balance explicitly. In our visual design, users control the data accuracy with the zoom level by scrolling and panning. Data queries are therefore designed to fetch specific data based on the zoom level and panning window. Although we didn't implement progressive updating, with this approach, it can render an initial view quickly with approximate or partial data, and then refine the visualization with more accurate data as computation and data transfer complete. This approach ensures that the user receives immediate feedback, even if initially approximate, while progressively improving the representation. Visualizing all data at once is rarely useful for exploratory analysis; instead, the goal should be to present data that is useful for exploration while maintaining interactive responsiveness.

When visual interfaces require multiple rounds of data fetches, prioritization becomes essential. Designers should identify which data attributes are most critical to support early exploration and retrieve those first, deferring additional details until user interaction signals their necessity. For example, Traveler first fetches event start and end times to provide a responsive overview of the timeline. Additional attributes, such as primitive names, are only retrieved upon user request.

Additionally, a multiple level of abstraction in the visual interfaces helps to navigate large data, maintaining both representativeness and interactivity. For example, data for the Aggregated Gantt View is only fetched when the user interacts with nodes in the Dependency Tree View. This staged strategy prioritizes interactivity while still supporting representativeness when needed. By adopting progressive refinement~\cite{jo2014livegantt} and multi-level abstraction with data prioritization strategies~\cite{battle2020structured}, visual designers can ensure that exploratory analysis remains useful. This reinforces the central thesis of the dissertation: connecting visual design considerations with data management strategies enables event sequence visualizations that remain both interactive and representative, even under the constraints of scale.

\subsection{Limitations of a Connected Approach}

While connecting visual design considerations with data management strategies brings substantial value, this approach has several limitations. One limitation is that a connected design often lead to highly tailored solutions, systems that are deeply optimized for a specific type of visualization or data structure. The considerations discussed in this section make a visualization system powerful within its intended domain but reduce its generalizability across different visualization contexts or datasets.

Another difficulty lies in anticipating what to prioritize between representativeness and interactivity. In some cases, maintaining high visual representativeness can impose significant data management overhead, reducing interactivity. In others, data constraints limit the level of visual detail that can be feasibly rendered. Balancing between representativeness and interactivity in a scalable manner is inherently complex and context-dependent.

These limitations highlight the necessity for careful consideration of the specific goals and use-cases of a visualization systems, the nature of the data being represented, and the needs of users to balance customization and generalizability.

\section{Visual Design Guidelines}

Based on the discussions in the previous section, we propose visual design guidelines in this section for creating useful event sequence visualizations by connecting visual design considerations with data management strategies.

\subsection{Use Connected Multi-Level abstraction}
\label{subsec:connected_multi_level_abstraction}

The first proposed design guideline for event sequence visualization is to use connected multi-level abstraction, where abstraction levels are equivalent across both the visual design and data management componenets. Large event datasets cannot be meaningfully explored without scalable mechanisms for shifting between overview and detail. Visual interfaces should therefore provide intermediate view abstraction, presenting data in various contexts and supporting user needs. The components that fetch data should be aware of these visual transitions and adjust the data abstraction accordingly. A common pitfall in visualization systems is to treat visual abstraction and data abstraction as separate concerns.

To address this, connected multi-level abstraction ensures that the visual levels of detail correspond directly to data abstraction for the data management. In ESeMan, for example, the hierarchical data structure is explicitly aligned with visual zoom levels. Each level in the hierarchy corresponds to the resolution and granularity visible at that scale. A connected abstraction ensures consistency between what appears on screen and how data is processed. The connection between abstractions also enhances efficiency by allowing the system to fetch only the data necessary for the current abstraction level. Additionally, a coupled abstraction helps designers maintain transparency by clearly indicating how each abstraction is derived, preserving representativeness and preventing users from mistaking summaries for complete data.

The concept of connected abstraction adopts ideas from extensible visualization frameworks~\cite{vieth2023manivault, heer2023mosaic, satyanarayan2016vega} that support seperate use-case specific visual design extensions. These frameworks require data transformations between abstraction levels that we attempt to eliminate through direct mappings between the abstractions for event sequence visualization specifically. By ensuring that abstraction levels are equivalent in both the visual and data domains, visualization designers can create systems that maintain coherence between what is visualized and how it is computed, guiding towards a useful, scalable event sequence visualization.

% \subsection{Show Multi-Level Abstractions}
% \label{subsec:multi_level_abstraction}

% A core guideline for event sequence visualization is to support \emph{multi-level abstractions}. Large event datasets cannot be meaningfully explored without scalable mechanisms for shifting between overview and detail. Visual interfaces should therefore provide intermediate view abstractions, presenting data in various contexts and supporting user needs. For example, Traveler presents the Dependency Tree View and the Aggregated Gantt View, which provide abstractions between the Utilization view and the Gantt View. 

% Transparency about the level of abstraction is also important. Users should be able to distinguish whether they are seeing raw events, aggregated summaries, or derived structures. Explicitly highlighting how an abstraction is obtained—whether through aggregation, clustering, or summarization—helps through annotations, labels, or other visual marks, which should preserve representativeness by preventing users from mistaking summaries for complete data.

% By showing multi-level abstractions and clarifying how they are constructed, visual systems can ensure that exploration remains both scalable and accurate to the underlying data. This guideline reflects the central thesis of this dissertation: that useful visualization requires co-design of visual representations and data management strategies. Hierarchical data structures, such as KD-trees, can be adopted for these multi-level visual abstractions, ensuring that users can navigate across levels of scale while maintaining interactive performance.

% \subsection{Allow Users to Control What is Being Visualized}
% \label{subsec:user_control}

% The second guideline comes from balancing between representativeness and interactivity. Achieving both simultaneously is challenging, but our findings suggest that they need not be opposing forces. Instead, they can be viewed as adjustable dimensions that the visualization system or the user can balance depending on the analytical context.

% Providing users with control over what is being visualized—whether summaries or details—serves as a practical mechanism for navigating this balance. Multi-level abstractions, combined with interaction techniques such as zooming, panning, and filtering, enable users to dynamically choose how much information to reveal at a time. For instance, a user may begin by viewing a high-level aggregated Gantt overview to identify trends, and then zoom into specific event sequences to inspect details. In Traveler, this flexibility is achieved through a shared query design that supports both summary and detailed data retrieval, allowing users to move smoothly between scales of representation without breaking interactivity.

% While an inverse relationship often exists between representativeness and interactivity, the appropriate balance depends on the use case. In some contexts, higher interactivity with reduced representativeness—such as visualizing streaming data averaged over longer time intervals—may be more useful. In others, greater representativeness with lower interactivity—such as rendering high-fidelity 3D models or complete execution traces—may be appropriate. The key insight is that visual systems should not fix this trade-off, but allow designers and users to adapt it to their analytical needs.

% This guideline exemplifies the central thesis of the dissertation: that useful event sequence visualizations arise from connecting visual design considerations with data management strategies. By allowing users to control the balance between summary and detail, systems can remain both responsive and representative, adapting to diverse analytical scenarios while maintaining usefulness.


% Throughout this dissertation, I have discussed the tension between \textit{representativeness}—faithfully displaying all events—and \textit{interactivity}—maintaining responsiveness. While these often appear opposed, they can be adjusted based on context. Visualization systems should allow users to control the level of summarization and detail to match their analytical goals. For instance, interactive summaries may suit streaming or exploratory views, whereas more representative but slower renderings benefit detailed analyses. Such user-steerable abstraction supports progressive disclosure \cite{Heer2012,Elmqvist2011} and promotes transparency about when data are summarized versus fully represented.


% \subsection{Allow Users to Control What is Being Visualized}
% \label{subsec:user_control}

% For a large number of events, provide users with explicit control over what is being visualized, particularly in terms of whether summaries or details are shown. Multi-level abstraction is one means of supporting this control, but interaction techniques such as zooming, panning, and filtering are equally important. By allowing users to decide when to remain at a high-level overview and when to reveal fine-grained details, the visualization supports flexible analysis paths that align with the user's intent.

% For event sequences, this control is critical because the same dataset may need to be examined at different levels of granularity. An overview of task utilization over time can guide hypothesis generation, while detailed event records support root cause analysis or anomaly detection. Traveler's utilization view and Gantt view illustrate how a shared query can provide both overview summaries and detailed event data, depending on user interaction. Similarly, zooming and panning along a timeline enable users to focus on areas of interest without being overwhelmed by irrelevant events.

% Providing users with control over what is visualized also helps maintain representativeness while ensuring interactivity. Summaries prevent the interface from becoming cluttered or unresponsive, while the option to reveal detail ensures that important events are not hidden.

\subsection{Prioritize Visual Tasks Before Optimizing Data Management}
\label{subsec:user_tasks}

Another guideline is to begin the visualization design process by identifying user goals and the visual tasks that serve those goals. Subsequently, data management strategies should be identified to support the visual tasks. Starting from user tasks provides a clear rationale for which queries must be efficient, what abstractions are needed, and where optimization efforts should be focused. Without this grounding, data management decisions risk over-optimizing for tasks that are rarely performed or under-supporting tasks that are critical to the analysis workflow.

For event sequence visualization, common user tasks include exploring event distributions over time, comparing utilization across resources, and examining dependencies among events. The Gantt Chart Task Taxonomy formalized many of these tasks, providing a structured framework for reasoning about task demands. Traveler exemplifies this guideline by aligning its multiple coordinated views with distinct task categories. Similarly, ESeMan tailored its query design to range and conditional range queries, as these directly map to the most frequent and useful Gantt tasks.

Prior visualization research~\cite{munzner2009nested, brehmer2013multi, munzner2014vad} has emphasized that design should start with user tasks rather than data structures or formats. Building on this foundation, this dissertation shows that identifying tasks not only guides visual design but also informs the choice of data management strategies. By optimizing queries that directly map to frequent tasks, our systems achieve interactivity while maintaining representativeness.

