\chapter{Conclusion and Future Work}
\label{chap:conclusion}

This dissertation investigated visual design and data management considerations for event sequence visualization. The central thesis is that useful event sequence visualization emerges from integrating the visual design and data management requirements as interdependent components. Through the design of Traveler, the proposed Gantt Chart Task Taxonomy, and the development of the ESeMan library, this work demonstrated how such integration can enable scalable, interactive, and representative visualization of event sequences.

\section{Summary of Contributions}

The first major contribution, Traveler, illustrated the benefits of a configurable, multi-view visualization platform for analyzing task-parallel execution traces. Developed in close collaboration with HPC researchers, Traveler demonstrated that responsiveness and adaptability are essential for sustaining meaningful analysis in rapidly evolving contexts.  

The second contribution, the Gantt Chart Task Taxonomy, provided a structured account of the diverse analytic tasks that Gantt charts support across domains. This taxonomy synthesized fragmented literature, clarified underexplored areas, and highlighted the need to evaluate visualization systems not solely on performance but also on their ability to accommodate diverse user tasks and reasoning strategies.  

The third contribution, ESeMan, addressed the data management side of the challenge by evaluating summarization and indexing strategies that enable interactive queries over large-scale event sequences. By systematically comparing approaches such as KD-trees and clustering, ESeMan revealed how data structures directly influence interactivity, memory efficiency, and visual accuracy.  

Together, these contributions provide a foundation for treating event sequence visualization as a coupled system of visual design considerations and data management strategies, ensuring that representations remain both interactive and accurate to the event data. The dissertation advances both methodological insights, in terms of design and task analysis, and technical strategies, in terms of data summarization and indexing.

\section{Reflections on the Dissertation}

A recurring theme across the three contributions is that the practical requirement for useful visualization is multifaceted. It is not only a matter of processing large data volumes but also of supporting the diversity of contextual operations that analysts must perform. Traveler showed that navigability on large data requires multiple levels of abstraction between overview and detail; the taxonomy highlighted queries needed to optimize based on visual tasks; and ESeMan enabled balancing between interactivity and visual accuracy, considering data management requirements. Taken together, these works underscore that the usefulness of a visualization system lies in its ability to balance the practical requirements by addressing relevant constraints together.

\section{Future Work}

Although this dissertation provides a coherent framework for integrating design and data management, it also opens several directions for further research.  

\subsection{Expanding Empirical Evaluation}
The Gantt Chart Task Taxonomy would benefit from extensive user studies across different application domains to validate the relative importance of tasks and to refine the taxonomy in light of practice. Such studies would highlight domain-specific priorities that may inform visualization design guidelines. Similarly, ESeMan should be evaluated with end users to better understand trade-offs between system efficiency and human interpretability. These evaluations are needed to ensure that the system’s computational gains translate into practical usability and analytic insight, ultimately strengthening the connection between data management strategies and practical exploratory analysis workflows.

\subsection{Adoption Across Domains}
The techniques demonstrated in this dissertation, while grounded in HPC and Gantt chart visualization, can be adopted in other domains with large event sequences, including healthcare monitoring, industrial process control, and digital activity analysis. Successful adaptation, however, requires tailoring data structures and visualization designs to the conventions and priorities of each field, providing a rich avenue for future exploration. Furthermore, this dissertation opens a potential research direction in identifying guidelines that do not prescribe a single solution, instead providing a toolkit of strategies for aligning tasks, visual representations, and data management techniques in domain-specific settings.

\subsection{Adaptive and Intelligent Data Management}
A promising direction is the development of adaptive systems that dynamically adjust data summarization strategies based on the user’s analytic context. For example, reinforcement learning~\cite{mao2019pensieve} could be used to optimize prefetching policies by learning which regions of the data a user is likely to explore next, while active learning~\cite{ren2021survey} could help prioritize which summaries to refine when computational resources are limited. Predictive models~\cite{canay2024predictive} trained on past interaction logs could anticipate zooming, panning, or filtering actions and prepare relevant summaries in advance, thereby reducing latency. Clustering-based methods~\cite{jain2010data} could further adapt granularity by automatically merging or splitting intervals depending on the density of events in the user’s current focus. Together, these approaches point toward intelligent data managers that continuously align summarization strategies with analytic behavior, enhancing both responsiveness and usability in event sequence visualization.

\subsection{Integrating Perceptual Awareness of Users}
Future visualization systems should also explore perceptual awareness that determines how many visual elements can be effectively perceived before the elements overwhelm the user. Integrating this knowledge into automated design choices would benefit the data management choices by ensuring that visualizations remain usable and interpretable even at scale. ESeMan’s results provide a starting point for connecting data management with perceptual limits, but more empirical studies are needed. Future research efforts can utilize ESeMan as the experimental platform and the contributed Gantt task taxonomy as a structured set of tasks. Such efforts would provide evidence for visualization design in guiding when and how to summarize, aggregate, or filter events. Establishing perceptual limits is crucial for designing systems that align with human-centered exploration, thereby enhancing both efficiency and analytical effectiveness.

\subsection{Integrating Perceptual Awareness of Users}
Future visualization systems should also explore perceptual awareness that determines how many visual elements can be effectively perceived before the elements overwhelm the user. Integrating this knowledge into automated design choices would benefit data management strategies by ensuring that visualizations remain usable and interpretable even at scale. ESeMan’s results provide a starting point for connecting data management with perceptual limits, but more empirical studies are needed. Future research efforts can utilize ESeMan as the experimental platform and the contributed Gantt task taxonomy as a structured set of tasks. Such studies would provide evidence-based thresholds for visualization design, guiding when and how to summarize, aggregate, or filter events. Establishing these perceptual limits is critical to designing systems that not only scale computationally but also align with human cognitive capacity, thereby improving both efficiency and analytic effectiveness.

\section{Concluding Remarks}
This dissertation demonstrated that useful event sequence visualization arises from the coupling of visual design considerations and data management techniques. By integrating these perspectives, it contributed new systems, taxonomies, and algorithms that advance the state of the art. More broadly, it offered a conceptual reframing of scalability in visualization, emphasizing that interactive and representative systems must be designed with both human and computational constraints in mind. The future directions outlined above provide opportunities to extend these insights, with the ultimate goal of enabling visualization systems that empower analysts across a wide range of data-rich domains.
