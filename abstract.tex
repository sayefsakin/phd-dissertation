%%% The text of this file should be about 350 words or fewer.
\abstract{
Exploratory visual analysis of event sequences is prevalent in domains such as manufacturing, healthcare, and computing to make decisions that are difficult to make from individual events. However, as the data size grows, visualizing a large number of events makes decision-making harder and the visual interface increasingly less useful. Specifically, there is a tension between designing a representative visual interface of the events and managing the large volume of events for the visual interface while maintaining interactivity. Addressing only one side of the tension often results in systems that are less useful.
\\
The aim of this dissertation is to ease this tension by treating the visual design considerations and data management techniques as interdependent components rather than separate concerns. The central thesis of this dissertation is that useful event sequence visualization requires connecting visual design considerations with data management strategies, and demonstrates that such an interconnected approach enables useful visualization that is both interactive and representative of the events.
\\
To substantiate the central thesis, this dissertation presents three main contributions that provide specific ways to connect visual design considerations with data management strategies by (1) understanding design considerations to visualize the vast scale differences in data that are useful, (2) identifying visual tasks and relevant data management techniques to achieve interactivity, and (3) determining the data management strategies that can ensure both interactivity and proper representation of large event sequence data. These contributions advance the design of event sequence visualization systems that balance interactivity and responsiveness, thereby improving the usefulness of exploratory visual analysis of event sequences.
}