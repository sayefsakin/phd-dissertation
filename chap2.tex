%%% -*-LaTeX-*-


\chapter{Background and Related Work}
\label{chap:background}

Event sequence visualization is a widely studied problem that intersects data visualization, visual analytics, process mining, and data management. Prior work spans visualization techniques, data management strategies, summarization and reduction techniques for handling and visualizing a large number of events. This chapter reviews the literature along these lines, highlighting how this dissertation builds upon and extends prior contributions.

We use the terminology introduced in \autoref{chap:intro}. Events are defined by start and end times, and optionally contain attributes. Events are grouped into tracks that partition the dataset by a relevant attribute (such as computational resource or patient). Tracks are represented in parallel timeline charts and Gantt charts. We emphasize on techniques that support durational events containing dependencies and grouped attributes. These types of events generally arise in domains where automated event logs are generated over time, such as in visual exploratory analysis of execution traces.

It is important to note that this dissertation does not attempt to review all of the event sequence visualizations. Many alternative idioms~\cite{yeshchenko2024survey}, such as matrix diagrams, process mining visualizations, or Sankey-style flows, are outside the scope of this dissertation. Likewise, specialized techniques for categorical event logs, symbolic sequence mining, or anomaly detection are not addressed here. Readers interested in these broader aspects of event sequence visualization are directed to surveys by Aigner et al.~\cite{aigner2023visualization} and Guo et al.~\cite{guo2021survey} which provide comprehensive coverage of broader time-oriented and event-sequence data more generally.

\section{From Parallel Timeline Charts to Scalable Event Sequence Visualization}
A parallel timeline chart represents multiple related event sequences across tracks, with time on one axis and tracks on the other. Events are typically encoded as bars spanning their duration, defined by start and end times. Tracks are commonly derived by partitioning the event set by an attribute such as computational resource, patient, or user~\cite{wilson2003gantt,jo2014livegantt,antweiler2022uncovering}. This idiom underlies Gantt charts, widely used in execution trace analysis~\cite{isaacs2014state,ezzati2017multi}, user activity logs, manufacturing, and medical workflows.

While straightforward in principle, scaling parallel timeline charts is challenging. Zooming and panning are required to maintain visibility, and sub-pixel events must either be omitted, aggregated, or rendered in reduced form~\cite{guo2021survey}. Event summarization helps manage visual clutter but risks obscuring detail. Davidson et al.~\cite{davidson2023qualitative} showed that aggregation and accuracy remain persistent concerns in distributed trace tools.

Commercial and open-source systems support event sequence analysis in varying capacities. Google’s Perfetto~\cite{Perfetto} relies on in-memory browser databases and sampling, offering high interactivity but limited scalability. Grafana~\cite{grafana} provides state timelines with pagination-based subsampling, but cannot generate multi-scale overview summaries. Other proprietary platforms~\cite{Tableau,Jaeger,Zipkin,AmazonXRay,Datadog,Lightstep,Flurry,NewRelic,Honeycomb,Elastic,powerbi} serve diverse markets but require substantial infrastructure investments, cloud dependencies, and vendor lock-in, restricting accessibility for researchers and small organizations.

Data scale has been a persistent issue in execution trace visualization. Besides responsiveness issues, the significant scale disparity between the overall trace length and individual events can make Gantt charts challenging to interpret, particularly when dependencies are involved. Several systems have proposed partial solutions to address the data scale. SyncTrace~\cite{Karran2013SyncTrace} introduced a track-centric view with multiple levels of detail. Ravel~\cite{isaacs2014combing} employed an idealized unit time axis to reveal dependency patterns. Haugen et al.~\cite{haugen2015visualizing} displayed event dependencies only within a selected interval.

Our literature-based Gantt chart taxonomy~\cite{sakin2024gantttaxonomy} formalizes how tasks involving duration, dependencies, and grouping differ from timestamped events. Unlike prior taxonomies that treat events independently, it identifies gaps between visual tasks and data management strategies, establishing Gantt charts and parallel timeline charts as a useful medium for event sequence visualization.

Additionally, our work on the event sequence manager aims to provide an open-source, lightweight solution that delivers accuracy and scalability without relying on proprietary infrastructure. By coupling data structures with visualization tasks, it avoids the trade-offs inherent in DBMS-driven or sampling-based commercial systems.

\section{Data Management Strategies for Event Sequence Visualization}
Database management systems (DBMSs) have been widely used to support event sequence visualization~\cite{bell2003paraprof,sun2021daisen,battle2016dynamic,agarwal2013blinkdb,heer2023mosaic}. Distributed databases extend this approach for large-scale or streaming data~\cite{moritz2015perfopticon,kaldor2017canopy,kruchten2022vegafusion,shen2023qevis,kesavan2020visual,khan2023web}. While powerful, these systems depend heavily on fast network communication, distributed query execution, caching, and progressive rendering. Latencies frequently exceed the recommended 100ms threshold for interactivity~\cite{kaldor2017canopy,kruchten2022vegafusion}, and configuration overhead adds further barriers.

Battle et al.~\cite{battle2020database} benchmarked databases for interactive visualization, noting DuckDB as a high-performing option. However, even efficient DBMSs incur overhead from query translation and data transfer. Battle et al.~\cite{battle2020structured} further classified hybrid solutions that combine DBMSs with indexing or sketching techniques.

In our first contribution, instead of using a DBMS, Traveler~\cite{sakin2022traveler} uses a summed area table for storing event sequences. During this work, we still experience lag for larger datasets. We identified that to overcome the lag, scalable exploration required tighter coupling between data handling and task design. Subsequently, our ESeMan library~\cite{sakin2025managingdatascalableinteractive} takes an alternative approach by combining KD-Tree indexing with LMDB~\cite{lmdb} for storage. These works bypass DBMS communication overhead, enabling scalable, interactive event sequence queries without complex distributed configuration.

\section{Techniques for Reducing Data Scale in Visualization}

Data summarization techniques are frequently employed to improve interactivity. Approaches such as binning~\cite{carr1987scatterplot,liu2013immens}, M4 optimizations for rasterization~\cite{jugel2014m4}, and materialized views in visualization grammars~\cite{satyanarayan2016vega,heer2023mosaic,battle2020structured} reduce query sizes but often require additional memory and may discard fine-grained detail.

Linear data structures like summed-area tables and sampling methods such as LTTB have been applied in visualization systems~\cite{van2022plotly,sakin2022traveler}. KD-Tree–based methods (e.g., Kyrix-J~\cite{tao2020kyrix}, KD-Box~\cite{zhao2021kd}) show promise for summarizing time-series visualization. Neural networks and clustering techniques~\cite{jo2014livegantt,nesi2023summarizing,pasupathi2021trend} provide domain-specific event summarization, though they are computationally intensive and sensitive to data distributions.

Nanocubes~\cite{lins2013nanocubes} trade off initial build time by storing specialized indexes for spatio-temporal queries. Multi-variate data-tiles~\cite{moritz2019falcon} are used to prefetch queries by modeling user behavior patterns. By compromising memory consumption, these methods improve visual interactivity in multi-scale zooming and panning.

\begin{sloppypar}
Our work ESeMan extends these techniques by applying KD-Trees to event sequence data, enabling summarization that maintains fidelity while supporting interactive retrieval. Compared with sketching~\cite{cormode2017data,budiu2015interacting} or clustering-based summarization, it prioritizes scalability and accuracy in post-hoc trace analysis.
\end{sloppypar}

\section{Summary}
Visualization techniques, data management strategies, summarization and reduction techniques each address parts of the event sequence visualization problem. However, existing solutions in the literature compromise visual accuracy, incur interactivity overhead, or require infrastructure that creates barriers to adoption by smaller organizations or individual researchers. This dissertation addresses these gaps through three complementary contributions: Traveler, which demonstrates task-driven navigation for HPC traces; the Gantt chart taxonomy, which formalizes unique visualization tasks for Gantt-style charts; and ESeMan, which proposes a scalable, fidelity-preserving data management technique. Together, these contributions demonstrate that effective event sequence visualization requires treating visual design and data management as interdependent concerns.